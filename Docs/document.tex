\documentclass{article}
\usepackage{fancyhdr}
\pagestyle{fancy}
\lhead{Progetto finale di Reti Logiche - A.A. 2019/2020}
\rhead{Weger Marco}
\cfoot{\thepage}

% (1) choose a font that is available as T1
% for example:
\usepackage{lmodern}
% (2) specify encoding
\usepackage[T1]{fontenc}
% (3) load symbol definitions
\usepackage{textcomp}

\usepackage[italian]{babel}

\addto\captionsitalian{% Replace "english" with the language you use
  \renewcommand{\contentsname}%
    {Contenuto}%
}

\usepackage{graphicx}

\begin{document}
\pagenumbering{gobble}
\title{Progetto finale di Reti Logiche}
\author{Weger Marco - Matricola n° 888201}
\date{Anno Accademico 2019/2020}
\maketitle

\tableofcontents

\newpage
\pagenumbering{arabic}
\section{Introduzione}
Per questo progetto mi sono posto l'obiettivo di descrivere un componente che rispetti le specifiche sia pre sintesi che in post sintesi.
Ho voluto scrivere del codice di facile lettura e che si adatti in modo semplice e rapido a qualsiasi tipo di modifica del pattern e/o della dimensione della memoria e del suo contenuto (più dettagli in seguito).
La FPGA consigliata non ci pone particolari vincoli di area quindi ho voluto dare particolare riguardo ai tempi di esecuzione.
Per quanto riguarda la frequenza di clock non sono andato alla ricerca di una massimizzazione in quanto la specifica fissa il periodo di clock a 100 ns.
\subsection{Obiettivi aggiuntivi}
Una volta rispettate le richieste della specifica fornita mi sono posto i seguenti obiettivi:
\begin{enumerate}
	\item Minimizzare il tempo trascorso dal momento che il segnale di start viene ricevuto al momento di invio del segnale di done;
	\item Disattivare il segnale di enable della memoria tra le varie esecuzioni;
	\item Rendere il componente adattabile ad un'eventuale modifica della lunghezza dell'indirizzo della cella di memoria tramite una costante;
	\item Rendere il componente adattabile ad un'eventuale modifica della dimensione di una singola cella di memoria tramite una costante (ADDR);
	\item Rendere il componente adattabile ad un'eventuale modifica del numero di elementi in una working-zone tramite una costante (WZ\textunderscore OFFSET);
	\item Rendere il componente adattabile ad un'eventuale modifica del numero di working-zone tramite una costante (WZ\textunderscore NUM).
\end{enumerate}

Al fine di raggiungere i sopracitati obiettivi ho assunto che l'indirizzo da codificare e l'indirizzo codificato vengano sempre salvati in successione in celle immediatamente consecutive all'ultimo indirizzo di working-zone (es. se ci fosse 16 working-zone RAM(16) conterrebbe l'indirizzo da codificare e RAM(17) l'indirizzo codificato).
Tutte le ottimizzazioni descritte in seguito sono state valutate sulla base dei dati forniti dalla specifica e non tengono conto dell'eventuale crescita sproposita delle costanti sopracitate.
\subsection{Funzionamento in sintesi}
Una soluzione che memorizza tramite registri i valori delle working-zone non avrebbe migliorato in modo significativo i tempi di esecuzione peggiorando però l'area occupata pertando ho optato per la  in memoria a ogni esecuzione.
La singola esecuzione di una codifica può essere descritta attraverso un numero finito di step (che poi diventeranno una macchina a stati finiti):
\begin{enumerate}
	\item Reset ed attesa del segnale di start (\textit{i\textunderscore start=1});
	\item Abilitazione della memoria e richiesta dell'indirizo da codificare (salvato in un registro);
	\item Richiesta della i-esima working-zone e confronto con l'indirizzo salvato, eventuale codifica e passaggio a step successivo (passo ripetuto per i compreso tra 0 e il numero di working-zone);
	\item Scrittura dell'indirizzo codificato in memoria;
	\item Invio segnale di elaborazione completata (\textit{o\textunderscore done=1}) e attesa feedback (\textit{i\textunderscore start=0}), il dato è disponibile fin dal momento in cui \textit{o\textunderscore done} viene portato a 1;
\end{enumerate}
\section{Architettura}
\section{Sintesi}
\section{Simulazioni}
\section{Conclusione}

\end{document}
